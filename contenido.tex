Alexander Von Humboldt

Alexander Von Humboldt, o Alejandro de Humboldt, fue un científico y viajero alemán que recorrió el territorio colombiano a principios del siglo XIX. Nace en Berlín, el 14 de septiembre de 1769. Alemania es entonces un Estado conformado por reinos en expansión territorial, por lo tanto en constantes conflictos fronterizos. Época de la Europa Napoleónica: la amenaza de invasión despierta la conciencia nacional, el modelo de La Ilustración resulta un medio efectivo para impulsar la literatura, el idioma y la cultura, como símbolos de la nacionalidad alemana, que para ese momento está mucho menos consolidada que la de Francia. Berlín se convierte en el principal centro cultural de Alemania. Allí esta Kant, Goethe, Mozart. Es el comienzo de los círculos intelectuales que integrarán los hermanos Humboldt, Guillermo y Alejandro.

Biografía

Alexander von Humboldt nació el 14 de septiembre en la ciudad prusiana de Tegel, hoy Alemania; falleció el 6 de mayo de 1859 en Berlín. Miembro de una familia acaudalada de origen prusiano, [1] hijo menor de María Elisabeth Colomb y Alexander Georg von Humboldt y hermano de Wilhelm von Humboldt. Su padre falleció en 1779 y su madre viuda se esforzó en que sus hijos recibieran educación de alto nivel.

Formación académica

Este es el contexto en que transcurre la vida de Humboldt. Su familia de origen prusiano había obtenido su título de nobleza pocos años antes de su nacimiento, situación que condiciona su formación y la de su hermano Wilhem y la orienta hacia un camino eminentemente político y militar, destino trazado por su familia y el estado. Antes de su ingreso a la universidad reciben una educación inmejorable a cargo de preceptores que los inician en las ciencias naturales, los idiomas y la literatura. Posteriormente Humboldt ingresa a la Universidad de Frankfurt de Oder para realizar algunos estudios de economía. Al año siguiente, en 1789, regresa a Berlín para prepararse en matemáticas, dibujo y grabado. Ese mismo año retoma sus estudios en la Universidad de Gotinga, asistiendo a cursos de geografía, botánica, geología y física. Durante este periodo realiza sus primeras exploraciones en Inglaterra y París. Humboldt nunca obtuvo un título universitario.

Se inicia como expedicionario

En 1791, Humboldt ingresa a la Escuela de Minas de Freiberg y llega a hacer parte del Consejo Superior de Minas. Cuando fallece su madre en 1796, abandona su cargo en el estado prusiano y recibe una herencia considerable que le permite llevar a cabo la expedición que silenciosamente había venido madurando desde sus primeros contactos con viajeros, botánicos y geólogos. Se dedica desde entonces a preparar su viaje a la América Equinoccial, realizando varias exploraciones a Jena, Viena, París y algunas ciudades españolas. Durante su estadía allí logra reunirse con Carlos IV, quien a través del Consejo de Indias le concede un pasaporte para viajar a América y así poder realizar con libertad todas sus investigaciones.

Una vez cristalizado su sueño de viajar, Humboldt se embarca junto a Aimé Bonpland, botánico y médico francés quien fuera su compañero durante toda la expedición y amigo por el resto de su vida, en la Fragata Pizarro. Los viajeros abandonan la embarcación en Venezuela debido a una epidemia de fiebre. Así a sus treinta años, el 16 de julio de 1799, Humboldt pisa tierras continentales de América en Cumaná. Iniciándose allí la primera exploración que deja atrás los tradicionales objetivos colonialistas y económicos que habían caracterizado a los viajeros que llegaban al Nuevo Continente.

Expedición científica por América

El trabajo de Humboldt y Bonpland se caracteriza por la dedicación académica y científica, que incluso supera los conocimientos propios de las disciplinas que los viajeros conocían: geomagnetismo, geografía, botánica y geología, por objetivo mayor, acercarse a una filosofía de la Tierra a través de un conocimiento íntegro de la Naturaleza, a partir de una ciencia autodidacta que no sólo funda sus cimientos en los saberes aprendidos sino que está dispuesta a crearse con las nuevas observaciones.
Es importante recordar que Humboldt jamás aceptó dineros de gobierno alguno para evitar que sus investigaciones se supeditaran a fines secundarios como el comercio y explotación de plantas y minerales, tal como ocurrió con la Expedición Botánica.

Luego de realizar algunas muestras y observaciones botánicas y geológicas durante más de cuatro meses en Cumaná, Caracas y otros lugares venezolanos, llegan el 27 de marzo de 1800 a San Fernando de Apure, establecimiento de las misiones capuchinas y punto de partida para el viaje fluvial por los ríos Orinoco, Apure, Atabapo y Negro, que lo llevará a puntos muy cercanos de la actual geografía colombiana y que puede contemplarse como la primera aproximación a sus estudios en los Andes, ya que cartografió ríos que como el Meta y el Arauca, nacen en la Alta Montaña colombiana. Estudia además ecosistemas afines a los territorios venezolanos y colombianos. Durante este viaje realiza sus primeros herbarios y algunas observaciones de fauna como la del gimnoto, los tigres, monos araguatos, cocodrilos y otras especies 
